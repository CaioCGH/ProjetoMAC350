
\documentclass{report}
\usepackage[utf8]{inputenc}
\usepackage[portuges]{babel}
\usepackage{fullpage}
\usepackage{graphicx}
\usepackage{tikz}
\usetikzlibrary{er,positioning}
\renewcommand{\baselinestretch}{0.5}
\author{Caio Calisto Gaede Hirakawa \\ Matheus Santos Conceição}
\title{Exercício-Programa 1: Base de dados}
\begin{document}
\maketitle
\tableofcontents
\chapter{Parte I: Entendimento e validação do modelo}
\paragraph{Validar o modelo}
\chapter{Parte II: Descrição das classes abstratas}

\section{Entidades}
\subsection{Pessoa:}
\begin{itemize}
  \item Número USP  (Chave primária)
  \item CPF (Chave secundária)
  \item Nome
  \item Idade
  \item Endereço:
  \begin{itemize}
  	\item Logradouro
  	\item Número
  	\item Complemento
  	\item Bairro
  	\item Município
  	\item Unidade Federativa
  	\item CEP
  	\end{itemize}
\end{itemize}
\begin{tikzpicture}
  \node[entity] (node1) {Pessoa}
  child [grow=-30,level distance=3cm]{node[attribute] {pe\_Name}}
  child [grow=-50,level distance=3cm]{node[attribute] {\underline{pe\_NUSP}}}
  child [grow=-90,level distance=3cm]{node[attribute] {\underline{\underline{pe\_CPF}}}}
  child [grow=0,level distance=3cm]{node[attribute] {pe\_Endereço}
    	child [grow=40,level distance=3cm]{node[attribute] {pe\_Numero}}
   		child [grow=16,level distance=4cm]{node[attribute] {pe\_Complemento}}
   		child [grow=5,level distance=3.5cm]{node[attribute] {pe\_Bairro}}
   		child [grow=-10,level distance=4cm]{node[attribute] {pe\_Municipio}}
   		child [grow=-25,level distance=4cm]{node[attribute] {pe\_Unidade\_Federativa}}
   		child [grow=-60,level distance=3cm]{node[attribute] {pe\_CEP}}
    };
\end{tikzpicture}

\paragraph{Restrições Entidade Regular:}
\begin{itemize}
  \item pe\_NUSP: Chave primária da entidade Pessoa. Sequência de 6 a 9 dígitos, englobando todos os números USP conhecidos e mais um dígito para quando os dígitos disponíveis não forem suficientes.
  \item CPF Chave secundária da entidade Pessoa. Sequência de 11 dígitos que passe pelas regras de validação da Receita Federal:
  \begin{itemize}
  	\item verificação do primeiro dígito
  	\item verificação de segundo dígito
  	\item verificação de dígitos iguais
  \end{itemize}
  \item pe\_Nome: Sequência de letras maiúsculas e minúsculas, ' e -. No mínimo de 5 e máximo de 80. 
  \item Data de Nascimento: no formato 
  \item Endereço:
  \begin{itemize}
  	\item Logradouro
  	\item Número
  	\item Complemento
  	\item Bairro
  	\item Município
  	\item Unidade Federativa
  	\item CEP
  	\end{itemize}
\end{itemize}

\subsection{Aluno:}
\begin{itemize}
  \item Curso*
\end{itemize}
\begin{tikzpicture}
  \node[entity] (node1) {Aluno}
  child [grow=-30,level distance=3cm]{node[attribute] {al\_Curso}};
\end{tikzpicture}

\footnotetext{* não pode ser nulo}

\subsection{Professor:}
\begin{itemize}
  \item Departamento
\end{itemize}
\begin{tikzpicture}
  \node[entity] (node1) {Professor}
  	child [grow=-30,level distance=3cm]{node[attribute] {pr\_Departamento}};
\end{tikzpicture}

\subsection{Administrador:}
\begin{itemize}
  \item
\end{itemize}
\begin{tikzpicture}
  \node[entity] (node1) {Administrador};
\end{tikzpicture}

\subsection{Disciplina:}
\begin{itemize}
  \item Nome
  \item Código
  \item Departamento
  \item Ementa
  \item Descrição
  \item Pré-requisitos
  
\end{itemize}
\begin{tikzpicture}
  \node[entity] (node1) {Disciplina}
  	child [grow=45,level distance=3cm]{node[attribute] {dis\_Nome}}
  	child [grow=25,level distance=3cm]{node[attribute] {\underline{dis\_Codigo}}}
  	child [grow=05,level distance=4cm]{node[attribute] {dis\_Departamento}}
  	child [grow=-10,level distance=3.5cm]{node[attribute] {dis\_Ementa}}
  	child [grow=-30,level distance=3cm]{node[attribute] {dis\_Descrição}}
  	child [grow=-60,level distance=3cm]{node[attribute] {dis\_Pre-requisitos}};
\end{tikzpicture}

\subsection{Curso:}
\begin{itemize}
  \item Nome
  \item Código
\end{itemize}
\begin{tikzpicture}
  \node[entity] (node1) {Curso}
  	child [grow=45,level distance=3cm]{node[attribute] {cur\_Nome}}
  	child [grow=25,level distance=3cm]{node[attribute] {\underline{cur\_Codigo}}};
\end{tikzpicture}

\subsection{Obrigatória:}
\begin{itemize}
  \item
\end{itemize}
\begin{tikzpicture}
  \node[entity] (node1) {Obrigatória};
\end{tikzpicture}

\subsection{Optativa:}
\begin{itemize}
  \item
\end{itemize}
\begin{tikzpicture}
  \node[entity] (node1) {Optativa};
\end{tikzpicture}

\subsection{Módulo:}
\begin{itemize}
  \item
\end{itemize}
\begin{tikzpicture}
  \node[entity] (node1) {Modulo};
\end{tikzpicture}

\subsection{Trilha:}
\begin{itemize}
  \item
\end{itemize}
\begin{tikzpicture}
  \node[entity] (node1) {Modulo};
\end{tikzpicture}

\subsection{Usuário:}
\begin{itemize}
  \item
\end{itemize}
\begin{tikzpicture}
  \node[entity] (node1) {Usuario};
\end{tikzpicture}

\subsection{Perfil:}
\begin{itemize}
  \item
\end{itemize}
\begin{tikzpicture}
  \node[entity] (node1) {Perfil};
\end{tikzpicture}

\subsection{Serviço:}
\begin{itemize}
  \item
\end{itemize}
\begin{tikzpicture}
  \node[entity] (node1) {Serviço};
\end{tikzpicture}

\subsection{Oferecimento:}
\begin{itemize}
  \item Semestre
\end{itemize}
\begin{tikzpicture}
  \node[entity] (node1) {Oferecimento}
  	child [grow=-45,level distance=3cm]{node[attribute] {of\_semestre}};
\end{tikzpicture}

\subsection{Grade:}
\begin{itemize}
  \item
\end{itemize}
\begin{tikzpicture}
  \node[entity] (node1) {Grade};
\end{tikzpicture}

%%%%%%%%%%%%%%%%%%%%%%%%%%%%%%%%%%%%%%%%%%%%%%%%%%%%%%%%%%%%%%%%%%%%%%%%%%
%         Relacionamentos (estou achando que não precisa dessa parte
%%%%%%%%%%%%%%%%%%%%%%%%%%%%%%%%%%%%%%%%%%%%%%%%%%%%%%%%%%%%%%%%%%%%%%%%%%%
\section{Relacionamentos}

\subsection{Ministra:}
\begin{itemize}
  \item
\end{itemize}
\begin{tikzpicture}
  \node[relationship] (node1) {Ministra};
\end{tikzpicture}

\subsection{Cursa:}
\begin{itemize}
  \item Data de Início
\end{itemize}
\begin{tikzpicture}
  \node[relationship] (node1) {Cursa}
  	child [grow=-45,level distance=3cm]{node[attribute] {Data\_Inicio}};
\end{tikzpicture}

\subsection{Planeja:}
\begin{itemize}
  \item Data de Inscrição
\end{itemize}
\begin{tikzpicture}
  \node[relationship] (node1) {Planeja}
  	child [grow=-45,level distance=3cm]{node[attribute] {Data\_Inscricao}};
\end{tikzpicture}

\subsection{Administra:}
\begin{itemize}
  \item
\end{itemize}
\begin{tikzpicture}
  \node[relationship] (node1) {Administra};
\end{tikzpicture}

\subsection{rel\_dis\_cur:}
\begin{itemize}
  \item
\end{itemize}
\begin{tikzpicture}
  \node[relationship] (node1) {rel\_dis\_cur};
\end{tikzpicture}

\subsection{op\_mod:}
\begin{itemize}
  \item
\end{itemize}
\begin{tikzpicture}
  \node[relationship] (node1) {op\_mod};
\end{tikzpicture}

\subsection{tr\_mod:}
\begin{itemize}
  \item
\end{itemize}
\begin{tikzpicture}
  \node[relationship] (node1) {tr\_mod};
\end{tikzpicture}

\subsection{us\_pf:}
\begin{itemize}
  \item
\end{itemize}
\begin{tikzpicture}
  \node[relationship] (node1) {us\_pf};
\end{tikzpicture}

\subsection{pf\_se:}
\begin{itemize}
  \item
\end{itemize}
\begin{tikzpicture}
  \node[relationship] (node1) {pf\_se};
\end{tikzpicture}
\end{document}
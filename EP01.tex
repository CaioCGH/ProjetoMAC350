
\documentclass{report}
\usepackage[utf8]{inputenc}
\usepackage[portuges]{babel}
\usepackage{fullpage}
\usepackage{graphicx}
\usepackage{tikz}
\usetikzlibrary{er,positioning}
\renewcommand{\baselinestretch}{0.5}
\author{Caio Calisto Gaede Hirakawa \\ Matheus Santos}
\title{Exercício-Programa 1: Base de dados}
\begin{document}
\maketitle
\tableofcontents
\chapter{Parte I: Entendimento e validação do modelo}
\paragraph{Validar o modelo}
\chapter{Parte II: Descrição das classes abstratas}

\section{Pessoa:}
\begin{itemize}
  \item Número USP  (Chave primária)
  \item CPF (Chave secundária)
  \item Nome*
  \item Idade*
  \item Endereço:
  \begin{itemize}
  	\item Logradouro*
  	\item Número*
  	\item Complemento
  	\item Bairro*
  	\item Município*
  	\item Unidade Federativa*
  	\item CEP*
  	\end{itemize}
\end{itemize}
\begin{tikzpicture}
  \node[entity] (node1) {Pessoa}
  child [grow=-30,level distance=3cm]{node[attribute] {pe\_Name}}
  child [grow=-50,level distance=3cm]{node[attribute] {\underline{pe\_NUSP}}}
  child [grow=-90,level distance=3cm]{node[attribute] {\underline{\underline{pe\_CPF}}}}
  child [grow=0,level distance=3cm]{node[attribute] {pe\_Endereço}
    	child [grow=40,level distance=3cm]{node[attribute] {pe\_Numero}}
   		child [grow=16,level distance=4cm]{node[attribute] {pe\_Complemento}}
   		child [grow=5,level distance=3.5cm]{node[attribute] {pe\_Bairro}}
   		child [grow=-10,level distance=4cm]{node[attribute] {pe\_Municipio}}
   		child [grow=-25,level distance=4cm]{node[attribute] {pe\_Unidade\_Federativa}}
   		child [grow=-60,level distance=3cm]{node[attribute] {pe\_CEP}}
    };
\end{tikzpicture}

\section{Aluno:}
\begin{itemize}
  \item Curso*
\end{itemize}
\begin{tikzpicture}
  \node[entity] (node1) {Aluno}
  child [grow=-30,level distance=3cm]{node[attribute] {al\_Curso}};
\end{tikzpicture}

\footnotetext{* não pode ser nulo}

\section{Professor:}
\begin{itemize}
  \item Departamento
\end{itemize}
\begin{tikzpicture}
  \node[entity] (node1) {Professor}
  	child [grow=-30,level distance=3cm]{node[attribute] {pr\_Departamento}};
\end{tikzpicture}

\section{Administrador:}
\begin{itemize}
  \item
\end{itemize}
\begin{tikzpicture}
  \node[entity] (node1) {Administrador};
\end{tikzpicture}

\section{Disciplina:}
\begin{itemize}
  \item Nome
  \item Código
  \item Departamento
  \item Ementa
  \item Descrição
  \item Pré-requisitos
  
\end{itemize}
\begin{tikzpicture}
  \node[entity] (node1) {Disciplina}
  	child [grow=45,level distance=3cm]{node[attribute] {dis\_Nome}}
  	child [grow=25,level distance=3cm]{node[attribute] {\underline{dis\_Codigo}}}
  	child [grow=05,level distance=4cm]{node[attribute] {dis\_Departamento}}
  	child [grow=-10,level distance=3.5cm]{node[attribute] {dis\_Ementa}}
  	child [grow=-30,level distance=3cm]{node[attribute] {dis\_Descrição}}
  	child [grow=-60,level distance=3cm]{node[attribute] {dis\_Pre-requisitos}};

\end{tikzpicture}


\end{document}


\documentclass{report}
\usepackage[utf8]{inputenc}
\usepackage[portuges]{babel}
\usepackage{fullpage}
\usepackage{graphicx}
\usepackage{tikz}
\usepackage{titlesec}
\usetikzlibrary{er,positioning}
\renewcommand{\baselinestretch}{0.5}


\author{Caio Calisto Gaede Hirakawa \\ Matheus Santos Conceição}
\title{Exercício-Programa 1: Base de dados}
\begin{document}
\maketitle
\tableofcontents
\chapter{Parte I: Entendimento e validação do modelo}
\paragraph{Validar o modelo}
\chapter{Parte II: Descrição das classes abstratas}

\section{Entidades}
\subsection{Pessoa:}
\begin{itemize}
  \item Número USP  (Chave primária)
  \item CPF (Chave secundária)
  \item Nome
  \item E-mail
  \item Data de Nascimento
  \item Endereço:
  \begin{itemize}
  	\item Logradouro
  	\item Número
  	\item Complemento
  	\item Bairro
  	\item Município
  	\item Unidade Federativa
  	\item CEP
  	\end{itemize}
\end{itemize}
\begin{tikzpicture}
  \node[entity] (node1) {Pessoa}
  child [grow=-125,level distance=3cm]{node[attribute] {pe\_Email}}
  child [grow=-30,level distance=3cm]{node[attribute] {pe\_Name}}
  child [grow=-50,level distance=3cm]{node[attribute] {\underline{pe\_NUSP}}}
  child [grow=-90,level distance=3cm]{node[attribute] {\underline{\underline{pe\_CPF}}}}
  child [grow=0,level distance=3cm]{node[attribute] {pe\_Endereço}
    	child [grow=40,level distance=3cm]{node[attribute] {pe\_Numero}}
   		child [grow=16,level distance=4cm]{node[attribute] {pe\_Complemento}}
   		child [grow=5,level distance=3.5cm]{node[attribute] {pe\_Bairro}}
   		child [grow=-10,level distance=4cm]{node[attribute] {pe\_Municipio}}
   		child [grow=-25,level distance=4cm]{node[attribute] {pe\_UF}}
   		child [grow=-60,level distance=3cm]{node[attribute] {pe\_CEP}}
    };
\end{tikzpicture}

\paragraph{Restrições Entidade Regular:}
\begin{itemize}
  \item \textbf{pe\_NUSP:} Chave primária da entidade Pessoa. Sequência de 6 a 9 dígitos, englobando todos os números USP conhecidos e mais um dígito para quando os dígitos disponíveis não forem suficientes. Não pode ser nulo.
  \item \textbf{pe\_CPF:} Chave secundária da entidade Pessoa. Sequência de 11 dígitos que passe pelas regras de validação da Receita Federal:
  \begin{itemize}
  	\item verificação do primeiro dígito
  	\item verificação de segundo dígito
  	\item verificação de dígitos iguais
  \end{itemize}
  Não pode ser nulo.
  \item \textbf{pe\_Nome:} Sequência de letras maiúsculas e minúsculas, apóstrofe, hífen e espaço. No mínimo de 5 e máximo de 80 caracteres. Não pode ser nulo.
  \item \textbf{pe\_Email:} Sequência de 1 a 80 caracteres alfanuméricos e caracteres especiais \begin{verbatim}(!#$%&'*+-/=?^_`{|}~.)\end{verbatim} seguidos de um @, seguido de mais uma sequência de 1 a 80 caracteres alfanuméricos e caracteres especiais \begin{verbatim}(!#$%&'*+-/=?^_`{|}~.)\end{verbatim}.
  Não pode ser nulo.
  \item \textbf{pe\_DataNascimento:} no formato data dd/mm/aaaa. Não pode ser nulo.
  \item \textbf{pe\_Endereco:}
  \begin{itemize}
  	\item \textbf{pe\_Logradouro:} Sequência de caracteres alfanuméricos, apóstrofe, hífen e espaço. No mínimo de 4 e máximo de 80 caracteres. Não pode ser nulo.
  	\item \textbf{pe\_Numero}: Sequência de 1 a 10 caracteres alfanuméricos e espaço, com o intuito de permitir escrever "sem número". Não pode ser nulo.
  	\item \textbf{pe\_Complemento:} Sequência de caracteres alfanuméricos, apóstrofe, hífen e espaço. No mínimo de 4 e máximo de 80 caracteres. Pode ser nulo.
  	\item \textbf{pe\_Bairro:} Sequência de caracteres alfanuméricos, apóstrofe, hífen e espaço. No mínimo de 4 e máximo de 80 caracteres. Não pode ser nulo.
  	\item \textbf{pe\_Municipio:} Complemento Logradouro: Sequência de caracteres alfanuméricos, apóstrofe, hífen e espaço. No mínimo de 4 e máximo de 80 caracteres. Não pode ser nulo.
  	\item \textbf{pe\_UF:} Complemento Logradouro: Sequência de caracteres alfanuméricos, apóstrofe, hífen e espaço. No mínimo de 4 e máximo de 80 caracteres. Não pode ser nulo.
  	\item \textbf{pe\_CEP:} Sequência de 5 dígitos, seguido de um hífen, seguidos de mais três dígitos. Não pode ser nulo.
  	\end{itemize}
\end{itemize}

\subsection{Aluno:}
\begin{itemize}
  \item Data de Ingresso
  \item Curso
  \item Créditos acumulados
  \begin{itemize}
  	\item Obrigatórios
  	\item Optativos Eletivos
  	\item Optativos Livres
  \end{itemize}

\end{itemize}
\begin{tikzpicture}
  \node[entity] (node1) {Aluno}
  child [grow=+10,level distance=3cm]{node[attribute] {al\_Curso}}
  child [grow=-90,level distance=1cm]{node[attribute] {al\_DataIngresso}}
  child [grow=-10,level distance=3cm]{node[attribute] {al\_Creditos}
  	child [grow=-20,level distance=3cm]{node[attribute] {al\_Obrigatorios}}
  	child [grow=-70,level distance=2cm]{node[attribute] {al\_Eletivos}}
  	child [grow=-135,level distance=2cm]{node[attribute] {al\_Livres}}
  };
  
\end{tikzpicture}
\paragraph{Restrições Entidade Regular:}
\begin{itemize}
  \item \textbf{Data de Ingresso:} Data no formato dd/mm/aaaa. Não pode ser nulo.
  \item \textbf{Curso:} Sequência de caracteres alfanuméricos, apóstrofe, hífen e espaço. No mínimo de 4 e máximo de 80 caracteres. Não pode ser nulo.
  \item \textbf{Créditos acumulados:}
  \begin{itemize}
  	\item \textbf{Obrigatórios:} inteiro entre 0 e +infinito. Não pode ser nulo.
  	\item \textbf{Optativos Eletivos:} inteiro entre 0 e +infinito. Não pode ser nulo.
  	\item \textbf{Optativos Livres:} inteiro entre 0 e +infinito. Não pode ser nulo.
  \end{itemize}
\end{itemize}

\subsection{Professor:}
\begin{itemize}
  \item Departamento
  \item Data de Admissão
\end{itemize}
\begin{tikzpicture}
  \node[entity] (node1) {Professor}
  	child [grow=-30,level distance=3cm]{node[attribute] {pr\_Departamento}}
  	child [grow=-80,level distance=2.5cm]{node[attribute] {pr\_DataAdmissao}};
\end{tikzpicture}
\paragraph{Restrições Entidade Regular:}
\begin{itemize}
  \item \textbf{pr\_Departamento:} Sequência de caracteres alfanuméricos, apóstrofe, hífen e espaço. No mínimo de 4 e máximo de 80 caracteres. Não pode ser nulo.
  \item \textbf{pr\_DataAdmissao:} Data no formato dd/mm/aaaa. Não pode ser nulo.
\end{itemize}

\subsection{Administrador:}
\begin{itemize}
  \item Data de Início da Gestão
  \item Data de Término da Gestão
\end{itemize}
\begin{tikzpicture}
  \node[entity] (node1) {Administrador}
  child [grow=-30,level distance=3cm]{node[attribute] {adm\_DataInicio}}
  child [grow=-80,level distance=2.5cm]{node[attribute] {adm\_DataTermino}};
\end{tikzpicture}
\paragraph{Restrições Entidade Regular:}
\begin{itemize}
  \item \textbf{adm\_DataInicio:} Data no formato dd/mm/aaaa. Não pode ser nulo.
  \item \textbf{adm\_DataTermino:} Data no formato dd/mm/aaaa. Não pode ser nulo.
\end{itemize}

\subsection{Disciplina:}
\begin{itemize}
  \item Nome
  \item Código (Chave Primária)
  \item Departamento
  \item Ementa
  \item Descrição
  \item Pré-requisitos
  \item Período ideal
  
\end{itemize}
\begin{tikzpicture}
  \node[entity] (node1) {Disciplina}
  	child [grow=45,level distance=3cm]{node[attribute] {dis\_Nome}}
  	child [grow=25,level distance=3cm]{node[attribute] {\underline{dis\_Codigo}}}
  	child [grow=05,level distance=4cm]{node[attribute] {dis\_Departamento}}
  	child [grow=-10,level distance=3.5cm]{node[attribute] {dis\_Ementa}}
  	child [grow=-30,level distance=3cm]{node[attribute] {dis\_Descrição}}
  	child [grow=-60,level distance=3cm]{node[attribute] {dis\_Pre-requisitos}}
  	child [grow=-130,level distance=4cm]{node[attribute] {dis\_PeríodoIdeal}};
\end{tikzpicture}
\paragraph{Restrições Entidade Regular:}
\begin{itemize}
  \item \textbf{dis\_Nome:} Sequência de caracteres alfanuméricos, apóstrofe, hífen e espaço. No mínimo de 4 e máximo de 80 caracteres. Não pode ser nulo.
  \item \textbf{dis\_Codigo:}  Sequência de caracteres alfanuméricos. No mínimo de 4 e máximo de 9 caracteres. Não pode ser nulo.
  \item \textbf{dis\_Departamento:} Sequência de caracteres alfanuméricos, apóstrofe, hífen e espaço. No mínimo de 4 e máximo de 80 caracteres. Não pode ser nulo.
  \item \textbf{dis\_Ementa:} Sequência de caracteres alfanuméricos, apóstrofe, hífen e espaço. No mínimo de 4 e máximo de 80 caracteres. Não pode ser nulo.
  \item \textbf{dis\_Descricao:} Sequência de caracteres UTF-8. Mínimo de 4 e máximo de 1000 caracteres. Não pode ser nulo.
  \item \textbf{dis\_PreRequisitos:} Sequência de caracteres alfanuméricos, apóstrofe, hífen e espaço.. Mínimo de 4 e máximo de 100 caracteres. Não pode ser nulo.
  \item \textbf{dis\_Período ideal:} Sequência de caracteres alfanuméricos
\end{itemize}

%%%%%%%%%%%%%%%%%%%%%%%%%%%%%%%%%%%%%%%%%%%%%%%%%%%%%%%%%%%%%%%%%%%%%%
%\paragraph{Especialização da Disciplina:}
%\begin{itemize}
%	\item \textbf{Obrigatória:}
%			\begin{itemize}
%  				\item \textbf{teste:}
%			\end{itemize}
%			\begin{tikzpicture}
%  				\node[entity] (node1) {Obrigatória};
%			\end{tikzpicture}
%			\paragraph{Restrições Entidade Regular:}
%			\begin{itemize}
%				\item \textbf{teste2:}
%			\end{itemize}
%			
%	\item \textbf{Eletiva:}
%	\item \textbf{Livre:} 
%\end{itemize}
%
%----------------------------------------------------------------------------------------------------------------------------------------------------------------------------------------
%Acho que não precisa colocar Obrigatoria/eletiva/livre dentro de Disciplina. Declara como Entidade né? Mesmo sendo especialização
%----------------------------------------------------------------------------------------------------------------------------------------------------------------------------------------
%
%%%%%%%%%%%%%%%%%%%%%%%%%%%%%%%%%%%%%%%%%%%%%%%%%%%%%%%%%%%%%%%%%%%%%%




%%%%%%%%%%%%%%%%%%%%%%%%%%%%%%%%%%%%%%%%%%%%%%%
% parei de inserir as restrições aqui, mas mesmo restrições anteriores devem
% precisar de uma revisão, principalmente no número max e min de carateres
%%%%%%%%%%%%%%%%%%%%%%%%%%%%%%%%%%%%%%%%%%%%%%%%%%%%%%%%%%%%%

\subsection{Currículo:}
\begin{itemize}
  \item Ano de Início (Chave primária)
\end{itemize}
\begin{tikzpicture}
  \node[entity] (node1) {Currículo}
  	child [grow=-90,level distance=1.5cm]{node[attribute] {\underline{cur\_AnoIni}}};
\end{tikzpicture}
\paragraph{Restrições Entidade Regular:}
\begin{itemize}
  \item \textbf{cur\_AnoIni:} Número inteiro de 4 dígitos entre 1500 e +infinito.
\end{itemize}

\subsection{Obrigatória:}
\begin{itemize}
  \item
\end{itemize}
\begin{tikzpicture}
  \node[entity] (node1) {Obrigatória};
\end{tikzpicture}

\subsection{Eletiva:}
\begin{itemize}
  \item
\end{itemize}
\begin{tikzpicture}
  \node[entity] (node1) {Eletiva};
\end{tikzpicture}

\subsection{Livre:}
\begin{itemize}
  \item
\end{itemize}
\begin{tikzpicture}
  \node[entity] (node1) {Livre};
\end{tikzpicture}

\subsection{Módulo:}
\begin{itemize}
  \item Nome
  \item Código (Chave primária)
\end{itemize}
\begin{tikzpicture}
  \node[entity] (node1) {Modulo}
  	child [grow=-30,level distance=3cm]{node[attribute] {mod\_Nome}}
  	child [grow=-80,level distance=3cm]{node[attribute] {\underline{mod\_Codigo}}};
\end{tikzpicture}
\paragraph{Restrições Entidade Regular:}
\begin{itemize}
  \item \textbf{mod\_Nome:} Sequência de caracteres alfanuméricos, apóstrofe, hífen e espaço. No mínimo de 4 e máximo de 80 caracteres. Não pode ser nulo.
  \item \textbf{mod\_Codigo:} Sequência de caracteres alfanuméricos. No mínimo de 4 e máximo de 9 caracteres. Não pode ser nulo.
\end{itemize}


\subsection{Trilha:}
\begin{itemize}
  \item Nome
  \item Código (Chave primária)
\end{itemize}
\begin{tikzpicture}
  \node[entity] (node1) {Trilha}
  	child [grow=-30,level distance=3cm]{node[attribute] {tr\_Nome}}
  	child [grow=-80,level distance=3cm]{node[attribute] {\underline{tr\_Codigo}}};
\end{tikzpicture}
\paragraph{Restrições Entidade Regular:}
\begin{itemize}
  \item \textbf{tr\_Nome:} Sequência de caracteres alfanuméricos, apóstrofe, hífen e espaço. No mínimo de 4 e máximo de 80 caracteres. Não pode ser nulo.
  \item \textbf{tr\_Codigo:} Sequência de caracteres alfanuméricos. No mínimo de 4 e máximo de 9 caracteres. Não pode ser nulo.
\end{itemize}

\subsection{Usuário:}
\begin{itemize}
  \item
\end{itemize}
\begin{tikzpicture}
  \node[entity] (node1) {Usuario};
\end{tikzpicture}

\subsection{Perfil:}
\begin{itemize}
  \item Tipo
  \item Data de Início
  \item Data de Término
\end{itemize}
\begin{tikzpicture}
  \node[entity] (node1) {Perfil}
  	child [grow=-20,level distance=3cm]{node[attribute] {pr\_Tipo}}
  	child [grow=-100,level distance=3cm]{node[attribute] {\underline{pr\_DataTerm}}}
	child [grow=-50,level distance=3cm]{node[attribute] {\underline{pr\_DataInicio}}};
\end{tikzpicture}
\paragraph{Restrições Entidade Regular:}
\begin{itemize}
  \item \textbf{pr\_Tipo:} Sequência de caracteres alfanuméricos. No mínimo de 4 e máximo de 80 caracteres. Não pode ser nulo.
  \item \textbf{pr\_DataInicio:} Data no formato dd/mm/aaaa. Não pode ser nulo.
  \item \textbf{pr\_DataTerm:} Data no formato dd/mm/aaaa. Não pode ser nulo.
\end{itemize}

\subsection{Serviço:}
\begin{itemize}
  \item Nome
\end{itemize}
\begin{tikzpicture}
  \node[entity] (node1) {Serviço}
  child [grow=-90,level distance=1.5cm]{node[attribute] {ser\_Nome}};
\end{tikzpicture}
\paragraph{Restrições Entidade Regular:}
\begin{itemize}
  \item \textbf{ser\_Nome:} Sequência de caracteres alfanuméricos. No mínimo de 4 e máximo de 80 caracteres. Não pode ser nulo.
\end{itemize}

\subsection{Oferecimento:}
\begin{itemize}
  \item Semestre
\end{itemize}
\begin{tikzpicture}
  \node[entity] (node1) {Oferecimento}
  	child [grow=-45,level distance=3cm]{node[attribute] {of\_semestre}};
\end{tikzpicture}

\subsection{Grade:}
\begin{itemize}
  \item
\end{itemize}
\begin{tikzpicture}
  \node[entity] (node1) {Grade}
  	child [grow=0,level distance=3cm]{node[attribute] {cur\_Nome}}
	child [grow=-20,level distance=3cm]{node[attribute] {\underline{cur\_Código}}};
\end{tikzpicture}
\paragraph{Restrições Entidade Regular:}
\begin{itemize}
  \item \textbf{Nome:} Sequência de caracteres alfanuméricos, apóstrofe, hífen e espaço. No mínimo de 4 e máximo de 80 caracteres. Não pode ser nulo.
  \item \textbf{Código:} Sequência de caracteres alfanuméricos
\end{itemize}
%%%%%%%%%%%%%%%%%%%%%%%%%%%%%%%%%%%%%%%%%%%%%%%%%%%%%%%%%%%%%%%%%%%%%%%%%%
%         Relacionamentos (estou achando que não precisa dessa parte
%%%%%%%%%%%%%%%%%%%%%%%%%%%%%%%%%%%%%%%%%%%%%%%%%%%%%%%%%%%%%%%%%%%%%%%%%%%
\section{Relacionamentos}

\subsection{Ministra:}
\begin{itemize}
  \item
\end{itemize}
\begin{tikzpicture}
  \node[relationship] (node1) {Ministra};
\end{tikzpicture}

\subsection{Cursa:}
\begin{itemize}
  \item Data de Início
\end{itemize}
\begin{tikzpicture}
  \node[relationship] (node1) {Cursa}
  	child [grow=-45,level distance=3cm]{node[attribute] {Data\_Inicio}};
\end{tikzpicture}

\subsection{Planeja:}
\begin{itemize}
  \item Data de Inscrição
\end{itemize}
\begin{tikzpicture}
  \node[relationship] (node1) {Planeja}
  	child [grow=-45,level distance=3cm]{node[attribute] {Data\_Inscricao}};
\end{tikzpicture}

\subsection{Administra:}
\begin{itemize}
  \item
\end{itemize}
\begin{tikzpicture}
  \node[relationship] (node1) {Administra};
\end{tikzpicture}

\subsection{rel\_dis\_cur:}
\begin{itemize}
  \item
\end{itemize}
\begin{tikzpicture}
  \node[relationship] (node1) {rel\_dis\_cur};
\end{tikzpicture}

\subsection{op\_mod:}
\begin{itemize}
  \item
\end{itemize}
\begin{tikzpicture}
  \node[relationship] (node1) {op\_mod};
\end{tikzpicture}

\subsection{tr\_mod:}
\begin{itemize}
  \item
\end{itemize}
\begin{tikzpicture}
  \node[relationship] (node1) {tr\_mod};
\end{tikzpicture}

\subsection{us\_pf:}
\begin{itemize}
  \item
\end{itemize}
\begin{tikzpicture}
  \node[relationship] (node1) {us\_pf};
\end{tikzpicture}

\subsection{pf\_se:}
\begin{itemize}
  \item
\end{itemize}
\begin{tikzpicture}
  \node[relationship] (node1) {pf\_se};
\end{tikzpicture}
\end{document}


\documentclass{report}
\usepackage[utf8]{inputenc}
\usepackage[portuges]{babel}
\usepackage{fullpage}
\usepackage{graphicx}
\usepackage{tikz}
\usetikzlibrary{er,positioning}
\renewcommand{\baselinestretch}{0.5}
\author{Caio Calisto Gaede Hirakawa \\ Matheus Santos}
\title{Exercício-Programa 1: Base de dados}
\begin{document}
\maketitle
\tableofcontents
\chapter{Parte I: Entendimento e validação do modelo}
\paragraph{Validar o modelo}
\chapter{Parte II: Descrição das classes abstratas}
\section{Aluno:}
\begin{itemize}
  \item Número USP  (Chave primária)
  \item CPF (Chave secundária)
  \item Nome*
  \item Idade*
  \item Curso*
  \item Endereço:
  \begin{itemize}
  	\item Logradouro*
  	\item Número*
  	\item Complemento
  	\item Bairro*
  	\item Município*
  	\item Unidade Federativa*
  	\item CEP*
  	\end{itemize}
\end{itemize}

\begin{tikzpicture}[auto,node distance=1.5cm]
  % Create an entity with ID node1, label "Fancy Node 1".
  % Default for children (ie. attributes) is to be a tree "growing up"
  % and having a distance of 3cm.
  %
  % 2 of these attributes do so, the 3rd's positioning is overridden.
  \node[entity] (node1) {Aluno}
    [grow=up,sibling distance=10cm]
    child [grow=down,level distance=1cm]{node[attribute] {\underline{al\_NUSP}}}
    child [grow=down,level distance=2cm]{node[attribute] {\underline{\underline{alCPF}}}}
    child [grow=down,level distance=3cm]{node[attribute] {al\_Name}}
    child [grow=down,level distance=4cm]{node[attribute] {al\_Endereço}}
    child [grow=right,level distance=3cm]{node[attribute] {al\_Logradouro}};
\end{tikzpicture}

\footnotetext{* não pode ser nulo}

\end{document}
